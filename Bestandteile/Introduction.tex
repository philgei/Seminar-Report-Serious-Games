\chapter{Introduction}
As stated by Ribeiro et al, the term game atmosphere is used to describe "a subtle but important, intangible, generally aesthetic quality in games that leads to emotional immersion."\cite{Ribeiro.2020}. He derived this definition from several game reviews that were described as atmospheric, with these reviews referring to games such as Metro: Last Light, Limbo, and Firewatch. 
Furthermore, in a presentation at the Games Developer Conference (GDC), game designer Greg Kasavin, who worked on games such as Bastion tried to break down the concept of game atmosphere into a few technical aspects: He notes that atmosphere consists of thematic cohesion, internal consistency, and specific details \cite{GDD}.\\

However, in the same paper, from which this definition originates, it was also said in the same course that the concept of atmosphere in games is not the subject of many academic papers or research efforts. And although companies are slowly publishing scientific papers on the subject, it is mostly left to the artistic vision of a game studio or a designer.\cite{Ribeiro.2020}.
\\
Although a clear definition and also many scientific works on the subject of game atmosphere are missing, factors can be derived which clearly contribute to the atmosphere: visual design, story and how it is written, as well as the sound and music design \cite{MDA}.
\\
Therefore, this seminar paper deals with exactly this topic and tackles the first aspect of game atmosphere: The question of how visual effects and the use of lighting can be applied to achieve a desired atmosphere in games. First, many fundamental mechanics are explained, which are necessary to understand how lighting works in real life. This basic knowledge is necessary to venture into the virtual world and see how these fundamental properties are applied there and thematically adapted for a desired atmosphere. Once an appropriate set of tools has been established, it will be considered how these principles can be applied to a serious game developed in the context of this seminar work.

\section{Motivation}
Game atmosphere is one of the most important aspects a game has to offer \cite{Ribeiro.2020}. In his article for the website "Gamasutra", the game developer Matthew Bentley summarized the feeling of a dense atmosphere very well and fittingly: "Atmosphere is the feeling that remains when the rest of the game shuts the hell up" \cite{Gamasutra}. So the goal is to use visual effects to create a feeling for the player that is long lasting and can be conveyed to them without the use of other gameplay elements. \\

To accomplish this, however, a basic understanding of lighting itself and how it works in the real world must first be established. Only then is it possible to understand techniques used by lighting designers in all kinds of media. In addition, it is then possible to make use of the varied literature that exists for real lighting, as this can be applied to a very large extent to the design of video games\cite{Niedenthal1404353}. 