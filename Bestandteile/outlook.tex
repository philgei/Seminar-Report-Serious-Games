\chapter{Conclusion and Discusion}
The aim of this work was to show how a desired atmosphere can be achieved through the use of visual effects. The topic came into being because it was clear that every game creates a certain atmosphere. However, it was not clear how exactly this atmosphere can be created and which techniques have been developed for this purpose. Likewise, the goal of this work was to create a scientific vocabulary as a basis for describing such techniques. 

Therefore, the first step was to describe how lighting works in the real life and how measured values can be derived to describe just that. Then, the process of how these qualities of lighting can be simulated in the virtual world was described. With this knowledge, some techniques were then discussed by which an atmosphere can be achieved in video games, but also in the film industry, especially in animated films. \\
We have seen that by using three-point-lighting dramaturgic effects can be created, and thus the atmosphere of a scene or a game can be strongly impacted. We also looked at some ways to describe and summarize the combination of these techniques. After that, we saw how they have been applied in examples such as animated films. Finally, many of these techniques were applied in the context of a serious game, which was developed in parallel to this seminar work. \\
In the course of the work, it became especially clear that a precise scientific definition and regulation for aspects of game design and especially the creation of a desired atmosphere are missing. Even the concept of atmosphere in games is very difficult to narrow down and there are many attempts at definitions, which have also found their way into the elaboration of this work. \\
The same can be observed in the documentation of techniques used to design video games. While there are video games on the market that deliberately create a desired atmosphere and know exactly how to do it, unfortunately, these approaches are very rarely recorded in the form of scientific papers. Therefore, it was inevitable to make use of the work of film and even photography. However, as seen in the last chapter, it is partly very easy and intuitive to apply these techniques to video games.\\
\newpage
Finally, as the author, I would like to have my seminar paper briefly revisited again and express some thoughts. At the beginning it was very difficult to find a starting point for the research. There were some websites and especially YouTube videos that dealt with this topic, but I could not draw any empirical benefit from them, which is why they served as inspiration for some approaches, but by no means as a source. Therefore, and for the reason that these very effects were used in the accompanying development of a serious game, I decided to dive into the basics of lighting and describe them in great detail. Thus, the goal and title of this work was to show basic visual effects for creating a desired atmosphere and to explain their use. In my opinion, this goal was achieved. \\\\

As a final outlook, it remains to mention that one can be very excited about the further development of the artistic aspects of a video game. Especially when these are finally put down on paper.